%compile with :
%pdflatex memory_allocation.tex && bibtex memory_allocation && pdflatex memory_allocation.tex && pdflatex memory_allocation.tex && (rm *.aux *.bbl *.bib *.blg *.log *.toc *.out)

% clean with :
%(rm *.aux *.bbl *.bib *.blg *.log *.toc *.out)

\documentclass{article}

%package loading
\usepackage[utf8]{inputenc}
\usepackage{amsmath,amssymb,amsfonts,epsfig,color,colortbl,xcolor}
\usepackage{array}
\usepackage{pstricks,graphicx}
\usepackage{algorithm}
\usepackage{algpseudocode}
\usepackage{listings}
\usepackage{multirow}
\usepackage{ifthen}
\usepackage{fullpage}
\usepackage{filecontents}
\usepackage[hidelinks]{hyperref}

%package and document configuration
\newcommand{\email}[1]{\href{mailto:#1}{\nolinkurl{#1}}}
\newcommand{\TODO}[0]{\textbf{\color{red}{TODO}}}
%% choix des profondeurs de section pour la table des matières
%% 2= subsection, 3=subsubsection
% \setcounter{secnumdepth}{2}  %% Avec un numero.
% \setcounter{tocdepth}{2}     %% Visibles dans la table des matieres

\definecolor{colKeys}{rgb}{0,0,1}
\definecolor{colIdentifier}{rgb}{0,0,0}
\definecolor{colComments}{rgb}{0,0.5,1}
\definecolor{colString}{rgb}{0.6,0.1,0.1}
\definecolor{c1}{RGB}{219,144,71}
\definecolor{c2}{RGB}{100,212,78}
\definecolor{c3}{RGB}{255,111,111}
\definecolor{c4}{RGB}{111,139,255}
\definecolor{gris}{gray}{0.45}

\lstset{%configuration de listings
	float=hbp,%
	language=C,
	basicstyle=\ttfamily\small, %
	identifierstyle=\color{colIdentifier}, %
	keywordstyle=\color{colKeys}, %
	stringstyle=\color{colString}, %
	commentstyle=\color{colComments}, %
	columns=flexible, %
	tabsize=3, %
	frame=TRBL, %
	extendedchars=true, %
	%showspaces=false, %
	showstringspaces=false, %
	numbers=none, %
	breaklines=true, %
	breakautoindent=true, %
	captionpos=b,%
	xrightmargin=0cm, %
	xleftmargin=0cm,
	mathescape=true
}

%Metadata
\title{Memory Allocation and Array Contraction using the Polyhedral Model in
GeCoS}
% \title{Array Contraction using the Polyhedral Model in GeCoS}
\author{Antoine Morvan, CAIRN Team,
IRISA, \email{antoine.morvan@irisa.fr}}

%embedded bibtex file
\begin{filecontents}{memory_allocation.bib}
@ARTICLE{feautrier1991dataflow,
  author = {Feautrier, P.},
  title = {{Dataflow Analysis of Array and Scalar References}},
  journal = {International Journal of Parallel Programming},
  year = {1991},
  volume = {20},
  pages = {23--53},
  number = {1},
  publisher = {Springer},
  timestamp = {2010.09.30}
}
@INPROCEEDINGS{benabderrahmane2010polyhedral,
  author = {Benabderrahmane, M. W. and Pouchet, L.-N. and Cohen, A. and Bastoul,
	C.},
  title = {{The Polyhedral Model is More Widely Applicable Than You Think}},
  booktitle = {Proceedings of Int. Conf. on Compiler Construction},
  year = {2010},
  pages = {283--303},
  organization = {Springer}
}
@INPROCEEDINGS{bondhugula2008pluto,
  author = {Bondhugula, U. and Hartono, A. and Ramanujam, J. and Sadayappan,
	P.},
  title = {{PLuTo: A Practical and Fully Automatic Polyhedral Program Optimization
	System}},
  booktitle = {Proceedings of the ACM SIGPLAN Conference on Programming Language
	Design and Implementation},
  year = {2008},
  address = {Tucson, AZ},
  month = jun,
  organization = {ACM}
}
@incollection{LefebvreFeautrier,
year={1997},
isbn={978-3-540-63440-9},
booktitle={Euro-Par'97 Parallel Processing},
volume={1300},
series={Lecture Notes in Computer Science},
editor={Lengauer, Christian and Griebl, Martin and Gorlatch, Sergei},
doi={10.1007/BFb0002757},
title = {{Optimizing Storage Size for Static Control Programs in Automatic
Parallelizers}}, 
url={http://dx.doi.org/10.1007/BFb0002757},
publisher={Springer Berlin Heidelberg},
author = {Lefebvre, V. and Feautrier, P.},
pages={356-363}
}
@INPROCEEDINGS{BeeAlias07,
  author = {Alias, C. and Baray, F. and Darte, A.},
  title = {{Bee+Cl@k: an Implementation of Lattice-Based Array Contraction in
	the Source-to-Source Translator ROSE}},
  booktitle = {LCTES},
  year = {2007},
  pages = {73--82}
}
@INPROCEEDINGS{SvenISL10,
  author = {Verdoolaege, S.},
  title = {{ISL: An Integer Set Library for the Polyhedral Model}},
  booktitle = {ICMS},
  year = {2010},
  pages = {299--302},
  ee = {http://dx.doi.org/10.1007/978-3-642-15582-6_49}
}
\end{filecontents}

%Document start
\begin{document}
\maketitle

\emph{Note: This document assumes the reader is familiar with optimizing
compilation using the polyhedral model. For newcomers, I recommend the OpenScop
User Guide~\footnote{\url{http://icps.u-strasbg.fr/~bastoul/development/openscop/docs/openscop.pdf}}.}


{\center\noindent\makebox[\linewidth]{\rule{0.7\paperwidth}{0.4pt}}}\tableofcontents
{\center\noindent\makebox[\linewidth]{\rule{0.7\paperwidth}{0.4pt}}}

\section{Introduction}

The polyhedral model allows complex loop transformations for optimizing
different performance criteria, such as parallelism or memory access
locality~\cite{feautrier1991dataflow, benabderrahmane2010polyhedral,
bondhugula2008pluto}, on a subset of programs known as ACL (Aff Control
Loops). Once a loop nest has been scheduled, it is also possible to minimize the
memory footprint required to store the intermediate results computed during the
scheduled loop nest execution. 

Minimizing the memory footprint can be decomposed in three steps :
\begin{enumerate}
  \item identifying which write accesses can be re-allocated;
  \item constructing the problem inputs (i.e., the \emph{conflict sets},
  see~\ref{sec:conflict_set});
  \item solving the problems, resulting in the new access functions and array
  sizes.
\end{enumerate}

The goal of this document is to synthesize the existing
works~\cite{LefebvreFeautrier, BeeAlias07}, and to provide an implementation
oriented explanation of the different problematics. The representation of all
objects in the polyhedral model is presented in the ISL~\cite{SvenISL10}
formalism.

\vspace{0.5cm}
{\bf Note : for the sake of simplicity, the copies of ISL objects are omitted.}

\subsection*{Identifying Intermediate Results}

Of course, because it would alter the program semantic, one cannot change the
allocation of the input values, nor the allocation of the output values (i.e.,
the liveout values). It is simple to determine what are the input values (array
dataflow analysis answers $\bot$). However, the ACL representation does not
support ``what happened before and what happens after'' the ACL under analysis.
It is therefore not possible to know what value will be \emph{liveout} at the
end of the ACL execution, and what value is only an intermediate result.

From a higher position, this information can be computed before extracting the
ACLs from an AST-like representation of programs, even though it is not
possible to compute it accurately in the general case. Therefore, in the
remaining of the document, we assume that the values that can be
reallocated are specified by annotations, either on the array that stores the
values, or the statement that produces the values.

\subsection*{Intuition -- Successive Modulo Summary}

The basic idea behind memory reallocation is to first build the set of all the
pairs of operations for which a conflict holds, namely the pairs of operations
whose value is alive at the same time. This set can be built exactly and is
called the \emph{conflict set}. 

Then, from this conflict set, one can build the set containing all the
differences between each side for each pair. This is the \emph{difference set}
representing the lexicographic distance between each operation which conflicts.
For instance, if iterations $(0,0)$ and $(1,0)$ produce a value that is alive at
the same time, $((0,0),(1,0))$ and $((1,0),(0,0))$ are in the conflict set (it
is symetric), and $(1,0)$ and $(-1,0)$ are in the difference set.

By computing the lexicographic maximum of the difference set (per dimension,
see~\ref{sec:successive_modulo}), we obtain the size of the array (each maximum
on a dimension defining an \emph{expansion degree}). The access function for
the new array is the scheduled one, modulo the expansion degree.

{\center\noindent\makebox[\linewidth]{\rule{0.7\paperwidth}{0.4pt}}}
\section{Problem Construction -- Conflict and Difference Set}

% \subsection{Array Contraction vs. Memory Allocation}

The problem of minimizing the memory footprint after scheduling can be seen from
two point of view, namely \emph{array contraction} and \emph{memory
(re)allocation}. They result in different problem constructions, but the same
solving technique can be used in both cases. We consider that there is only one
write per statement (section~\ref{sec:multiwrite} gives hint for cases where
several writes occur per statement).

\subsubsection*{Array Contraction}

One way of seing the problem is to consider that the transformation has to keep
array names. In such a case, the schedule usually respects the memory
dependencies, and the original memory allocation remains legal. Minimizing the
memory footprint then consists in contracting existing arrays, and is called
\emph{array contraction}.

However, the given schedule may not respect the memory dependencies (write after
read and/or write after write), but only the dataflow dependencies (read after
write). Therefore, the original memory allocation may not be legal for the
schedule, and the minimum required memory size can be larger than the original
memory allocation. Finding this minimum size in such a case is called
\emph{partial array expansion}. It is simply a variant of array contraction
where the result may be larger than the original memory allocation.

\subsubsection*{Memory Allocation}

The other point of view is to consider that the memory will be completely
reallocated. The starting point is a \emph{fully expanded program} in single
assignment, namely the result of array dataflow
analysis~\cite{feautrier1991dataflow}. From there, original array names are
completely ignored, and the problem simply consists in minimizing the memory
footprint.

\subsection{Input Definition}
\label{sec:conflict_set}

The following subsections explain how to build the conflict set for both array
contraction and memory allocation, starting from the PRDG given by the array
dataflow analysis, and various information comming from the AST:
\begin{itemize}
  \item The PRDG: a graph where nodes represent statements, with their
  associated iteration domains, and edges represent dependencies between
  statements, with their associated relation and validity domain. This PRDG can
  be represented as a {\tt isl\_union\_map} and a {\tt isl\_union\_set} in ISL
  (cf. example).
  \item The writes: a relation that associates, for each statement, what are the
  written arrays, and for what index function. It can be represented using a
  single {\tt isl\_union\_map}.
  \item The reads: idem, but for reads.
  \item The schedule: new logical dates for the ACL operations. Represented as a
  single {\tt isl\_union\_map}, where the image is a single {\tt isl\_set}
  (all the ranges belong to the same name space).
\end{itemize}

\subsubsection*{Example 1 : Gaussian filter}

\begin{figure}[ht!]
\begin{lstlisting}
void gauss(int M, int N, int in[N][M], int out[N][M]) {
  int i, j;
  int tmp[N][M];
  for (i = 1; i < N - 1; i++)
    for (j = 0; j < M; j++)
S0:   tmp[i][j] = 3 * in[i][j] - in[i - 1][j] - in[i + 1][j];
  for (i = 1; i < N - 1; i++)
    for (j = 1; j < M - 1; j++)
S1:   out[i][j] = 3 * tmp[i][j] - tmp[i][j - 1] - tmp[i][j + 1];
}
\end{lstlisting}
\caption{
Example ACL, representing a $3 \times 3$ Gaussian filter
\label{fig:example_acl1}}
\end{figure}
The ACL from Figure~\ref{fig:example_acl1} leads to the following objects in the
polyhedral model (ISL notation, details of construction skipped):
\begin{itemize}
  \item Domains: $[N, M] \rightarrow \{ $\\ 
	  $S0[0,i,j] : 0 \le j \le M-1 ~\&~ 1 \le i \le N-2;$\\
	  $S1[1,i,j] : 1 \le j \le M-2 ~\&~ 1 \le i \le N-2 \}$
  \item PRDG: $[N, M] \rightarrow \{$\\
	  $S1[1,i,j] \leftrightarrow S0[0,i,j'] : j-1 \le j' \le j+1 \}$\\
	  \emph{Note : it is a relation.}
  \item writes: $[N, M] \rightarrow \{$\\
	  $S0[0,i,j] \rightarrow tmp[i, j];$\\
	  $S1[1,i,j] \rightarrow out[i, j] \}$
  \item reads: $[N, M] \rightarrow \{$\\
	  $S0[0,i,j] \rightarrow in[i-1, j]; S0[0,i, j] \rightarrow in[i, j]; S0[0,i,
	  j] \rightarrow in[i+1, j];$\\
	  $S1[1,i,j] \rightarrow tmp[i, j-1]; S1[1,i, j] \rightarrow tmp[i, j]; S1[1,i,
	  j] \rightarrow tmp[i, j+1] \}$
\end{itemize}
On top, we use the following schedule (partial loop fusion) :
\begin{itemize}
  \item schedule: $[N, M] \rightarrow \{$\\
	  $S0[0,i,j] \rightarrow [i, j];$\\
	  $S1[1,i,j] \rightarrow [i, j+2] \}$
\end{itemize}

\subsubsection*{Example 2 : Forward Substitution} 

\begin{figure}[ht!]
\begin{lstlisting}
void forward_subst(int N, int b[N], int L[N][N], int y[N]) {
  int i, j;
  for(i = 0; i < N; i++) {
S0: y[i] = b[i];
    for(j = 0; j < i; j++)
S1:   y[i] = y[i] - L[i][j] * y[j];
S2: y[i] = y[i] / L[i][i];
  }
}
\end{lstlisting}
\caption{
Example ACL, representing a forward substitution on square system matrix
\label{fig:example_acl2}}
\end{figure}
The ACL from Figure~\ref{fig:example_acl2} leads to the following objects in the
polyhedral model (ISL notation, details of construction skipped):
\begin{itemize}
\item Domains: $[N] \rightarrow \{$\\ 
 $S0[i,0,0] : i >= 0 ~\&~ i <= -1 + N;$\\
 $S1[i,1,j] : j >= 0 ~\&~ j <= -1 + i ~\&~ i >=0 ~\&~ i <= -1 + N; $\\
 $S2[i,2,0] : i >= 0 ~\&~ i <= -1 + N \}$
\item PRDG: $[N] \rightarrow \{$\\
 $S1[i,1,0] \rightarrow S0[i,0,0];$\\
 $S1[i,1,j] \rightarrow S1[i,1,j-1] : j >= 1;$\\
 $S1[i,1,j] \rightarrow S2[j,2,0];$\\
 $S2[0,2,0] \rightarrow S0[0,0,0];$\\
 $S2[i,2,0] \rightarrow S1[i,1,i-1] : i >= 1 \}$
\item writes: $[N] \rightarrow \{ $\\
 $S0[i,0,0] \rightarrow y[i];$\\
 $S1[i,1,j] \rightarrow y[i];$\\
 $S2[i,2,0] \rightarrow y[i] \}$
\item reads: $[N] \rightarrow \{$\\
 $S0[i,0,0] \rightarrow b[i];$\\
 $S1[i,1,j] \rightarrow L[i, j]; S1[i,1,j] \rightarrow y[i]; S1[i,1,j] \rightarrow y[j];$\\
 $S2[i,2,0] \rightarrow L[i, i]; S2[i,2,0] \rightarrow y[i] \}$
\end{itemize}
On top, we use the following schedule (identity) :
\begin{itemize}
  \item schedule: $[N] \rightarrow \{ $\\
 $S0[i,0,0] \rightarrow S0[i,0,0];$\\
 $S1[i,1,j] \rightarrow S1[i,1,j];$\\
 $S2[i,2,0] \rightarrow S2[i,2,0] \}$
\end{itemize}

\subsection{Dependencies and Domain Selection}

The first step consists in selecting dependencies and domains related to the
statement/array that needs to be reallocated. Basically it sums up to selecting
nodes and edges in the PRDG.

\emph{Note : special attention is required in the case where several writes can
occur in a single statement, see~\ref{sec:multiwrite}}.

\subsubsection{Selection for Memory Reallocation}

For the memory allocation, the analysis is computed per statement.

\TODO

The resulting objects are:
\begin{itemize}
  \item $PRDG_S$ is a subset of the PRDG, containing only the dependencies where
  the statement $S$ is the producer;
  \item $D_S$ is the iteration domain of $S$.
\end{itemize}

\TODO : example

\subsubsection{Selection for Array Contraction}

For the array contraction, the analysis is computed per array. For contracting
an array {\tt t}, the analysis has to focus on the dependencies involving {\tt
t} only. Selecting the dependencies involving {\tt t} only can be done by
filtering the PRDG :
\begin{enumerate}
\item build the list of the statements reading {\tt t} : $lreads$ (test the
tuple name)
\item build the list of the statements writing {\tt t} : $lwrites$ (test the
tuple name)
\item $PRDG_t = PRDG \cap lreads \times lwrites$
\item $D_t = lwrites$
\end{enumerate}

The resulting objects are:
\begin{itemize}
  \item $PRDG_t$ is a subset of the PRDG, containing only the dependencies
  involving the array {\tt t};
  \item $D_t$ is the iteration domains of all statements writing in {\tt t}.
\end{itemize}

\TODO : example

\subsection{Conflict Set Construction}

Once the dependencies have been filtered for the particular statement/array, it
is possible to compute the \emph{conflict set},
namely the set of all pairs of operations in the scheduled ACL for which there
is a \emph{conflict}. A conflict holds iff, given two existing operations
$\vec{w_1}$ and $\vec{w_2}$:
$$CS = \vec{w_1} \bowtie \vec{w_2} : \{ [\vec{w_1}] \rightarrow [\vec{w_2}] :
\vec{w_1} \preceq lastuse(\vec{w_2})~\&~\vec{w_2} \preceq lastuse(\vec{w_1})
\}$$ that is if the last use of the value produced by $\vec{w_2}$ is after
$\vec{w_1}$, and the last use of the value produced by $\vec{w_1}$ is after
$\vec{w_2}$.

\subsubsection{Working in the Scheduled Space}
Starting from the previously filtered PRDG (either $PRDG_S$ or $PRDG_t$, noted
$PRDG_f$) and domain (either $D_S$ or $D_t$, noted $D_f$) the first step
consists in representing $PRDG_f$ and $D_f$ %, writes and reads
in the scheduled space. This will make the analysis valid given the schedule:
\begin{enumerate}
\item $scheduledPRDG_f = schedule \circ PRDG_f$\\
{\tt isl\_union\_map\_apply\_range(isl\_union\_map\_apply\_domain($PRDG_f$,schedule),schedule)}
\item $scheduledD_f = schedule[D_f]$\\
{\tt isl\_union\_set\_apply($D_f$,schedule)}
% \item $scheduledWrites = schedule \circ writes$\\
% {\tt isl\_union\_map\_apply\_domain(writes,schedule)}
% \item $scheduledReads = schedule \circ reads$\\
% {\tt isl\_union\_map\_apply\_domain(reads,schedule)}
\end{enumerate}

The scheduled space unifies all the statements. Therefore, the scheduled object
above represent {\tt map / set} instead of {\tt union\_map /
union\_set}\footnote{Use functions isl\_union\_set\_get\_set\_at($scheduledD_f$,0) /
isl\_union\_map\_get\_map\_at($scheduledPRDG_f$,0)}. Also, let $n$ be the number of
dimensions in the scheduled space.

\subsubsection{Deriving the Conflict Set}
\label{sec:derive_cs}
In the scheduled space, the conflict set $CS$ (built as a relation here) is
derived as follows:
\begin{enumerate}
  \item $lastUse = lexMax(scheduledPRDG_f^{-1})$\\
  {\tt isl\_map\_lexmax(isl\_map\_reverse($scheduledPRDG_f$))}\\
  The function $lastUse$ associates, to a write, its last use (read) in the
  scheduled PRDG.
  \item $lexGE = \{ \vec{x} \leftrightarrow \vec{y}~|~ \vec{x} \succeq \vec{y}
  \}$ with $lexGE \subseteq \mathbb{Z}^n \times \mathbb{Z}^n$\\
  {\tt isl\_map\_lex\_ge(isl\_set\_get\_space($D_f$))}\\
  This relation represents the lexicographic predecessors.
  \item $predWrites = lexGE \cap \mathbb{Z}^n \times ran(scheduledPRDG_f)$\\
  {\tt isl\_map\_intersect\_range($lexGE$,isl\_map\_range($scheduledPRDG_f$))}\\
  This relation associates, to a read, all the writes preceding it.
  \item $CS_{part1} = predWrites \circ lastUse$\\
  {\tt isl\_map\_apply\_range($lastUse$,$predWrites$)}\\
  This relation associates, to a write, all the writes executed before its last
  use.
  \item $CS_{part2} = CS_{part1} \cap CS_{part1}^{-1}$\\
  {\tt isl\_map\_intersect($CS_{part1}$,isl\_map\_reverse($CS_{part1}$))}
  \item Since $CS$ is symetric, any write conflicts with itself:\\
  $CS = CS_{part2} \cup (\{ \vec{x} \leftrightarrow \vec{y}~|~\vec{x} =
  \vec{y}\} \cap D_f \times D_f)$\\
  {\tt tmp1 = isl\_map\_lex\_eq(isl\_set\_get\_space($D_f$))}\\
  {\tt tmp2 = isl\_map\_intersect\_domain(isl\_map\_intersect\_range(tmp,$D_f$),$D_f$)}\\
  {\tt isl\_map\_union($CS_{part2}$,tmp2)}
\end{enumerate}

In order to build a set ($\subseteq \mathbb{Z}^{2n}$) out of this relation: \\
{\tt
isl\_map\_get\_range(isl\_map\_move\_dims($CS$,isl\_dim\_out,0,isl\_dim\_in,0,$n$))}


%cs.moveDims(JNIISLDimType.isl_dim_out, 0, JNIISLDimType.isl_dim_in, 0, nbDims);

% 
% 
% {\center\noindent\makebox[\linewidth]{\rule{0.7\paperwidth}{0.4pt}}}
% 
% \begin{enumerate}
% \item $PRDG = \{\vec{x} \leftrightarrow \vec{y}~|~\vec{x} \in \mathcal{D} \wedge
% \vec{y} \in \mathcal{D} \wedge \vec{y} \in d[\{\vec{x}\}] \}$\\
% $PRDG = \{\vec{x} \leftrightarrow \vec{y}~|~\vec{x} \in \mathcal{D} \wedge
% \vec{y} \in \mathcal{D} \wedge \vec{x} \in d^{-1}[\{\vec{y}\}] \}$
% \item $PRDG_{min} = \{\vec{x} \rightarrow \vec{y}~|~\vec{x} \in \mathcal{D}
% \wedge \vec{y} \in \mathcal{D} \wedge \vec{y} = min(d[\{\vec{x}\}]) \}$\\
% $PRDG_{min} = \{\vec{x} \rightarrow \vec{y}~|~\vec{x} \in \mathcal{D}
% \wedge \vec{y} \in \mathcal{D} \wedge \vec{x} = max(d^{-1}[\{\vec{y}\}]) \}$
% \item $CS_1 = ( \{ \vec{y} \leftrightarrow \vec{z}~|~ \vec{y} \succeq \vec{z}
% \}~\circ~ PRDG_{min} ) \cap \{ \vec{x} \leftrightarrow \vec{z}~|~\vec{z}
% \in \mathcal{D}\}$\\ 
% $CS_1 = \{\vec{x} \leftrightarrow \vec{z}~|~\vec{x} \in
% \mathcal{D} \wedge \vec{z} \in \mathcal{D} \wedge
% min(d[\{\vec{x}\}]) \succeq \vec{z} \}$\\ 
% $CS_1 = \{\vec{x} \leftrightarrow \vec{z}~|~\vec{x} \in
% \mathcal{D} \wedge \vec{z} \in \mathcal{D} \wedge
% max(d^{-1}[\{\vec{z}\}]) \succeq \vec{x} \}$
% \item $PRDG^{-1} = \{\vec{x} \leftrightarrow \vec{y}~|~\vec{x} \in \mathcal{D}
% \wedge \vec{y} \in \mathcal{D} \wedge \vec{y} \in d^{-1}[\{\vec{x}\}] \}$
% \item $PRDG_{max}^{-1} = \{\vec{x} \rightarrow \vec{y}~|~\vec{x} \in
% \mathcal{D} \wedge \vec{y} \in \mathcal{D} \wedge \vec{y} =
% max(d^{-1}[\{\vec{x}\}]) \}$
% \item $CS_2 = ( \{ \vec{y} \leftrightarrow \vec{z}~|~ \vec{y} \succeq \vec{z}
% \}~\circ~ PRDG_{max}^{-1} ) \cap \{ \vec{x} \leftrightarrow \vec{z}~|~\vec{z} \in
% \mathcal{D}\}$\\
% $CS_2 = \{\vec{x} \leftrightarrow \vec{z}~|~\vec{x} \in
% \mathcal{D} \wedge \vec{z} \in \mathcal{D} \wedge
% max(d^{-1}[\{\vec{x}\}]) \succeq \vec{z} \}$
% \item $CS = CS_1 \cap CS_2$\\
% $CS = \{\vec{x} \leftrightarrow \vec{z}~|~\vec{x} \in
% \mathcal{D} \wedge \vec{z} \in \mathcal{D} \wedge max(d^{-1}[\{\vec{x}\}])
% \succeq \vec{z} \wedge max(d^{-1}[\{\vec{z}\}]) \succeq \vec{x} \}$
% \end{enumerate}

\subsection{Difference Set Construction}

The difference set represents the set of differences of elements in the conflict
set:
$$DS = \{ \vec{u} ~|~ \vec{u} = \vec{x} - \vec{y} \wedge (\vec{x},\vec{y}) \in
CS \}$$

It is built as follows:
\begin{enumerate}
  \item $DS_{r1} = \{ CS \leftrightarrow \mathbb{Z}^n \}$\\
{\tt tmp = isl\_map\_move\_dims($CS$, isl\_dim\_in, 0, isl\_dim\_out, 0, $n$)\\
isl\_map\_insert\_dims(tmp, isl\_dim\_out, 0, $n$)}
  \item $sub = \{ (\vec{x},\vec{y}) \rightarrow \vec{x} - \vec{y} \} \subseteq
  (\mathbb{Z}^n \times \mathbb{Z}^n) \times \mathbb{Z}^n$\\
{\tt 
sub = isl\_basic\_map\_universe(isl\_map\_get\_space($DS\_{r1}$))\\
lspace = isl\_local\_space\_from\_space(isl\_map\_get\_space($DS\_{r1}$))\\
for ($i : 0 \rightarrow n$)\\
.\hspace{0.7cm}cst = isl\_equality\_alloc(lspace)\\
.\hspace{0.7cm}cst = isl\_constraint\_set\_constant\_si(cst,0);\\
.\hspace{0.7cm}cst = isl\_constraint\_set\_coefficient\_si(cst,isl\_dim\_in, i, 1);\\
.\hspace{0.7cm}cst = isl\_constraint\_set\_coefficient\_si(cst,isl\_dim\_in, i+$n$, -1);\\
.\hspace{0.7cm}cst = isl\_constraint\_set\_coefficient\_si(cst,isl\_dim\_out, i, -1);\\
.\hspace{0.7cm}sub = isl\_basic\_map\_add\_constraint(sub, cst);\\
}
  \item $DS_{r2} = DS_{r1} \cap sub$\\
{\tt isl\_map\_intersect($DS_{r1}$, sub)}
  \item $DS = ran(DS_{r2})$\\
{\tt isl\_map\_range($DS_{r2}$)}
\end{enumerate}


% \begin{enumerate}
% % \item $CS^{+} = \{ \vec{x} \leftrightarrow \vec{z}~|~ \vec{x} \succ \vec{z}
% % \}~\circ~ CS$\\
% % $CS^{+} = \{\vec{x} \leftrightarrow \vec{z}~|~\vec{x} \in
% % \mathcal{D} \wedge \vec{z} \in \mathcal{D} \wedge max(d^{-1}[\{\vec{x}\}])
% % \succeq \vec{z} \wedge max(d^{-1}[\{\vec{z}\}]) \succeq \vec{x} \wedge \vec{x}
% % \succ \vec{z} \}$
% \item $DS_R = \{ (\vec{x} \leftrightarrow \vec{z}) \rightarrow \vec{u}
% ~|~\vec{x} \in \mathcal{D} \wedge \vec{z} \in \mathcal{D} \wedge max(d^{-1}[\{\vec{x}\}])
% \succeq \vec{z} \wedge max(d^{-1}[\{\vec{z}\}]) \succeq \vec{x} \wedge \vec{x}
% \succ \vec{z} \wedge \vec{u} = \vec{x} - \vec{y} \}$
% \item $DS = \{ \vec{u}~|~ \exists (\vec{x},\vec{y}) \wedge \vec{x} \in
% \mathcal{D} \wedge \vec{z} \in \mathcal{D} \wedge max(d^{-1}[\{\vec{x}\}])
% \succeq \vec{z} \wedge max(d^{-1}[\{\vec{z}\}]) \succeq \vec{x} \wedge \vec{x}
% \succ \vec{z} \wedge \vec{u} = \vec{x} - \vec{y} \}$
% \end{enumerate}

\subsection{Special Cases}

\subsubsection{Parallel Schedules}

In the step 6 of the conflict set derivation (cf sub
section~\ref{sec:derive_cs}), instead of building the lexicographic equiality
function, apply equality only for non parallel dimensions. Apply no constraint
on parallel dimensions.

This means that, on a parallel dimension, all writes conflict with each other.

\subsubsection{Multiple Writes per Statement}
\label{sec:multiwrite}

The polyhedral model allows the representation of multiple writes per statement.
This case occurs, for instance, when using pure functions\footnote{Pure
functions do not have any side effect.} in statements. It is a particular case
where the above techniques does not apply, because the isolated dependencies
may be wrong. Let us examine the example loop nest of
Figure~\ref{fig:example_multiwrites}.
\begin{figure}[ht!]
\begin{lstlisting}
//Pure function : without side effect
void pureFunction(int* out1, int* out2, int in1, int in2) {
	*out1 = foo(in1, in2); //other pure
	*out2 = bar(in1, in2); //functions
}
void main() {
  ...
  for (i = 1 .. N)
    for (j = 0 .. N)
   /* the semantic we use assume the  
    * derefenced locations are written */
S:    pureFunction(&x[i][j],&y[i][j],x[i-1][j],y[i-2][j]);
   /* another notation would be */
S:    (x[i][j],y[i][j]) = pureFunction(x[i-1][j],y[i-2][j]);
  ...
}
\end{lstlisting}
\caption{
Example of an ACL involving pure functions where multiple writes occurs in one
statement.
\label{fig:example_multiwrites}}
\end{figure}
For such a loop nest, the dependencies per array are: 
\begin{itemize}
  \item on array {\tt x} : $d_1 : [N] \rightarrow \{ S[i,j] \rightarrow S[i-1,j]
  \}$
  \item on array {\tt y} : $d_2 : [N] \rightarrow \{ S[i,j] \rightarrow S[i-2,j]
  \}$
\end{itemize}
The PRDG for this example would have only one edge with an associated relation
instead of a function:
$$d : [N] \rightarrow \{ S[i,j] \leftrightarrow S[i',j] : i-2 \le i' \le i-1
\}$$
Indeed, using this relation for the {\tt y} array would lead to the correct
conflict set, because the lexicographic maximum of $d$ is $d_2$. However, it is
obvious that this same maximum, when considering array {\tt x}, results in an
overestimated distance, although conservative.

Computing accurate distances when statements may issue several writes requires a
more precise representation of the dependencies, where edges are associated per
read/write access instead of per statement (GeCoS provides such information
through the {\tt org.polymodel.dataflow} meta-model).


{\center\noindent\makebox[\linewidth]{\rule{0.7\paperwidth}{0.4pt}}}
\section{Successive Modulo Technique}
\subsection{Basic Principle}
\label{sec:successive_modulo}
\subsection{Extensions}

Let us consider the memory allocation for the fully expanded program, and
assigning one array per write. If a single read access in the original program
uses a value that can be produced by differents writes, the new memory
allocation implies guarded accesses. Example \ldots TODO

{\center\noindent\makebox[\linewidth]{\rule{0.7\paperwidth}{0.4pt}}}
\section{Other Approches}
See \emph{Lattice Based Memory Allocation}~\cite{BeeAlias07}.

\appendix

{\center\noindent\makebox[\linewidth]{\rule{0.7\paperwidth}{0.4pt}}}
\section{Partial Unrolling and Distributed Memory}

{\center\noindent\makebox[\linewidth]{\rule{0.7\paperwidth}{0.4pt}}}
\bibliographystyle{unsrt}
\bibliography{memory_allocation}

\end{document}
%Document end